\documentclass{beamer}
\usepackage{beamerthemesplit}
\usetheme{CambridgeUS}%\usetheme{Frankfurt}
%\usetheme{Laly}
\usepackage{verbatim}
\usepackage{multirow}
\usepackage{amsmath} 
\usepackage{colortbl} 
\usepackage[noend]{algorithmic}
\usepackage{algorithm}
\usepackage{epsfig}
\usepackage{color}
\usepackage{hyperref}
%\definecolor{pacificblue}{RGB}{59,110,143}
\newcommand{\blue}[1]{\textcolor{pacificblue}{\textbf{#1}}}
\newcommand{\todo}[1]{\textcolor{red}{\textbf{#1}}}
\newcommand{\myemph}[1]{{\it #1}}
\newcommand{\mybf}[1]{{\bf #1}}
\fontfamily{georgia}\selectfont\normalsize
\newcommand{\ghline}[0]{\arrayrulecolor[rgb]{0.635,0.635,0.635} \hline}
\newcommand{\tspace}{\rule{0pt}{2.6ex}}
%==================================================================================================================================

\setbeamertemplate{footline}[frame number]
\title{Texture Synthesis for Material Recognition\\
\normalsize{Master\rq s Thesis in Articial Intelligence --- Intelligent Systems}}
\author{Jasper van Turnhout\\Student no. 0312649\\jturnhou@science.uva.nl}
\date{November 15, 2011}
\begin{document}
\frame{\titlepage}
\section[Outline]{}
\frame{\tableofcontents}

%==================================================================================================================================
\section{Introduction}
\frame{
	\frametitle{Material Recognition}
}
\frame{
	\frametitle{Texture Synthesis}
}
\frame{
	\frametitle{Goal of this thesis}
}
%==================================================================================================================================
\section{Related Work}
\frame{
	\frametitle{Textons \& Filter Banks}
}
\frame{
	\frametitle{Multivariate Gaussian Distributions}
}
\frame{
	\frametitle{Minimal Training Images}
}

%==================================================================================================================================
\section{Approach}
\frame{
	\frametitle{Photometric Stereo}
}
\frame{
	\frametitle{PhoTex Database}
}
\frame{
	\frametitle{Generation of novel data}
}

%==================================================================================================================================
\section{Fundamentals}
\frame{
	\frametitle{Local Reflection}
}
\frame{
	\frametitle{Lambert's Cosine Law}
}

%==================================================================================================================================
\section{Reflection Models}
\frame{
	\frametitle{Lambertian}
}
\frame{
	\frametitle{Phong}
}
\frame{
	\frametitle{Blinn-Phong}
}
\frame{
	\frametitle{Cook-Torrance}
}
\frame{
	\frametitle{Oren-Nayar}
}
%==================================================================================================================================
\section{Experiments}
\frame{
	\frametitle{Two datasets}
}
\frame{
	\frametitle{Experiment A}
}
\frame{
	\frametitle{Experiment B}
}
\frame{
	\frametitle{Results}
}

%==================================================================================================================================
\section{Conclusion}
\frame{
	\frametitle{Conclusion}
}


%==================================================================================================================================
\end{document}
%==================================================================================================================================




