\hypertarget{conclusion}{
}

Some points for conclusion:
\begin{itemize}
	\item Synthesizing data is a great way to augment training data to improve classification rates
	\item Different reflection models applied to the problem give small variation in classification rates
	\item Although a minor improvement, the more physical reflection models for speculars improved over the emperical models. (Blinn-Phong and Cook-Torrance vs. Lambertian and Phong)
	\item Although physically a better approach, Oren-Nayar performed similar or worse than baseline Lambertian. 
	\item The above issue can be explained by the Lambertian assumption in the Photometric Stereo procedure
	\item With the current available datasets for PS, mainly PhoTex, its difficult to do more in-depth analysis of modelling the texture (surface normals, albedo) when only intensity information is present. 
	\item Additional analysis of the materials of the database could be of use for better estimation of parameters (such as the refraction indices for Fresnel and surface roughness).
	\item Additional color information could improve 1. the recovery of surface albedo and normals. 2. the quality of synthesized images, since these are directly dependent on 1.
\end{itemize}
